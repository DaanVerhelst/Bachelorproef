\chapter{Implementatie}
\label{ch:implementatie}

\paragraph{Inleiding}

Nu dat we voldoende voorkennis hebben omtrent zowel hardware als software alsook de meest efficiënte hardware en software bepaald hebben voor deze toepassing (nl. reddingsacties op diepzee) kunnen we verder gaan naar de implementatiefase van dit concept. Voordat we een goed voorstel kunnen geven voor de implementatie van drones bij reddingsacties moeten we uiteraard een goed beeld hebben van de huidige situatie binnen de zeevaartsector. Daarom zullen we ons in dit onderdeel gaan verdiepen in de bestaande procedures die bedrijven en reddingsdiensten moeten volgen alsook de wetten rond het gebruik van drones op zee. Daarna zullen we, met alle vooraf vergaarde kennis, de effectieve acties en implementatie van de drone gaan vastleggen. We bespreken dan alle stappen van opstijgen en landen tot het vastleggen van acties die de drone zou ondernemen in het geval van het detecteren van een drenkeling. 

\section{Bestaande procedures, wetten en verwerking}

\paragraph{Wetten rond dronegebruik op zee}

Op dit moment is het bijzonder moeilijk om dit concept in België te gaan implementeren. Door de KB RPAS wet (sinds 2016) zijn de enige instanties die onmiddellijk een drone mogen inzetten indien zij het nodig achten de politie, de brandweer en de civiele bescherming van België. Alle andere instanties zijn verplicht om heel wat papieren in te vullen en een aanvraag in te dienen. Op 1 januari 2021 zal er echter een wetgeving doorgevoerd worden op Europees niveau waardoor, indien er een gecertificeerde piloot aanwezig is, een drone ingezet mag worden zonder al dat papierwerk in te moeten vullen. Dan kunnen we dit concept wel implementeren, althans in België. 

Alle oceanen en zeeën zijn opgedeeld in SRRs of Search and Rescue Regions. Voor België is dit de Maritieme Search and Rescue Region (MSRR). Ook is het luchtruim opgedeeld in FIRs of Flight Information Regions. Voor het implementeren van dit concept moeten we beide zones dus in rekening brengen. Dit wil dus ook zeggen dat de drone enkel ingezet kan worden indien ze zich in wateren bevinden die dit, volgens de wetgeving van het respectievelijke land, toestaan. Voor het implementeren van dit concept zal dus een Europese of wereldwijde overeenkomst tot stand gebracht moeten worden met meerdere landen die ook interesse hebben in dit concept.   

\paragraph{Procedures van Reddingsdiensten}

Eens de reddingsdiensten een oproep binnenkrijgen in verband met een drenkeling en voldoende informatie vergaard hebben zullen zij vertrekken naar de plaats van het incident. Eens de reddingsdiensten daar aangekomen zijn zullen zij, indien het slachtoffer nog niet gevonden is, een zoekactie uitvoeren met behulp van boten en/of helikopters en aan de hand van een drift- of zoekplan. Ook staan zij in voor het coördineren en synchroniseren van andere reddingseenheden die eventueel ook ingezet werden. Eens de drenkeling gevonden is en vervolgens in veiligheid gebracht is, moeten de reddingsdiensten deze informatie doorgeven aan de juiste instantie. Voor België is dit bijvoorbeeld het Maritieme Reddings- en Coördinatiecentrum (MRCC). 

\paragraph{Procedures van bedrijven in de scheepvaartsector}

Bij een man-over-boord incident mag het personeel van het vaartuig zelf onmiddellijk actie ondernemen. Zij moeten niet wachten op de reddingsdiensten om zelf een poging te wagen, de drenkeling te redden. Het kan wel zijn dat de SAR mission coördinatoren (SAR staat voor Search And Rescue), de poging van het moederschip stop zet indien het verderzetten ervan inefficiënt geacht wordt.

\paragraph{Verwerking van drones in het bestaande systeem}

De eerste stap van de implementatie van dit concept zal inhouden dat, op Europees of wereldwijd vlak, zoveel mogelijk landen een overeenkomst verwezelijken dat dit concept toelaat. Indien een Belgisch schip zich in Spaanse wateren begeeft, dan moet Spanje het inzetten van de drone toelaten zonder dat er om gevraagd moet worden om zo snel mogelijk de drenkeling te vinden en te volgen. Anders verliest de drone opnieuw zijn waarde.
Eens dit gerealiseerd is, moeten we een gemeenschappelijk netwerk opstellen dat alle betrokken partijen in staat stelt om vlot met elkaar te communiceren. Eens de drone opstijgt, moet een signaal automatisch gegeven worden aan de reddingsdienst in kwestie, waarna deze dienst meteen contact kan opnemen met het moederschip waartoe de drone behoort, om verder te kunnen communiceren. Eens de reddingsdiensten bijna op de plaats van het incident aangekomen zijn, kan de gecertifieerde piloot, de drone ver genoeg verwijderen van de drenkeling zodat het niet in de weg van de reddingsdiensten vliegt. Dit kan ook verwerkt worden in de software van de drone zodat deze zich automatisch op een veilige afstand plaatst. Indien de drenkeling nog niet gevonden werd, kunnen de reddingsdiensten advies geven in verband met zoekpatronen die de crew kan volgen of vragen de drone terug te trekken zodat zij zelf de zoektocht kunnen verderzetten.

\section{Opstijgen en landen op zee}

In normale omstandigheden zou het landen en opstijgen met een drone geen overdreven moeilijke onderneming mogen zijn. Je plaatst de drone op een vlak oppervlak en de drone kan verticaal opstijgen. Gelijkaardig voor het landen kan je simpelweg de drone naar de landingsbaan of het landingsplatform sturen en dan de hoogte van de drone geleidelijk aan verlagen. Opstijgen en landen op een schip is echter niet zo gemakkelijk. Terwijl je op land een stilstaand platform hebt, is een schip constant in beweging. We willen uiteraard niet dat de drone tegen het schip of het materiaal op het schip botst. 

\paragraph{Vertrek}

Voor het vertrek moeten we dus rekening houden met de snelheid van het schip alsook de koers die het schip volgt want als we verticaal zouden opstijgen, is het niet onmogelijk dat de nog steeds varende boot, de stationaire drone onderschept. Hoewel het schip waarschijnlijk wel gestopt zal worden bij een man-over-boord incident, staat een groot schip niet meteen stil. We willen echter wel zo snel mogelijk de drone inzetten zodat de drenkeling zo snel mogelijk gevonden en gevolgd kan worden. Daarom moet de drone ergens kunnen opstijgen waar het geen obstakels kan tegenkomen waartegen het zou kunnen vliegen. Voorbeelden hiervan zijn aan de rand van het schip, op een verhoogd punt zoals een balkon of helemaal achteraan het schip. Deze locaties zijn uiteraard verschillend voor elk schip aangezien schepen veel verschillende vormen aan kunnen nemen.

\paragraph{Recovery}

Eens de drone zijn opdracht voltooid heeft, kunnen we de drone uiteraard niet gewoon achterlaten. We willen deze terug zien te krijgen zodat het bij het volgende incident opnieuw gebruikt kan worden. We kunnen enerzijds een "return to home" zender implementeren zodat de drone ten allen tijde weet waar het terecht kan na de reddingsactie of een gecertifieerd piloot kan het manueel terug vliegen. Deze zender geeft zijn locatie door aan de drone zodat de drone steeds de weg naar het schip terug kan vinden. We zitten echter wel opnieuw met het probleem dat een schip nooit volledig stil staat. Dit kan landen moeilijker maken, zeker als het schip nog aan vaarsnelheid voortbeweegt maar dit is vrij onwaarschijnlijk. Een minimum aan beweging van het schip is cruciaal.

\section{Acties bij detectie van een drenkeling}

Wat moet de drone nu doen wanneer het de locatie van de drenkeling ontdekt heeft. Er zijn verscheidene zaken die de drone kan en/of moet doen. De drone moet aangeven dat het iets of iemand gedetecteerd heeft. Er kan geopteerd worden voor het landen van de drone, met opblaasbare voetstukken, om de batterij van de drone te sparen tot de reddingsdiensten arriveren of de drone kan in de lucht blijven. Tot slot kan de drone gebruikt worden om de kalmte van de drenkeling te bewaren zodat hij of zij niet te veel energie verdoet en het zo langer kan uithouden tot de reddingsdiensten aankomen. Deze zaken zullen we nu bespreken.

\paragraph{Signalisatie van locatie}

Het is belangrijk dat, wanneer de drenkeling gevonden is, de drone zijn locatie doorgeeft aan zowel de crew van het schip als de reddingsdiensten. De crew van het schip zal ook al kunnen zien op de monitor wanneer de drone stil komt te staan. Dat zou één methode kunnen zijn voor het bepalen van de locatie van de drenkeling maar dit is natuurlijk niet de beste oplossing die we met deze technologie kunnen bereiken. Het automatisch signaleren en locatie delen zou een veel effectievere oplossing zijn. Zo kunnen de reddingsdiensten meteen naar de locatie van het signaal navigeren met een hoge waarschijnlijkheid, de drenkeling daar terug te vinden.  

\paragraph{Landen versus vliegen}

Eens de drone de locatie van de drenkeling ontdekt heeft en het signaal gegeven heeft zodat de reddingsdiensten ook op de hoogte gebracht zijn, kan de drone 2 zaken doen: landen op zee of in de lucht blijven. Beide opties hebben voordelen en nadelen die in rekening gebracht moeten worden. In de lucht blijven heeft als voordeel dat het veel gemakkelijker is om de drenkeling te blijven volgen om zo een accurate locatie door te kunnen blijven geven. Een groot nadeel is dat het  meer batterij zal verbruiken en als de reddingsdiensten niet op tijd aankomen zou het rampzalig zijn als de drone vroegtijdig zou moeten terug keren. Ook moeten we rekening houden met het feit dat reddingsacties dikwijls ondernomen worden aan de hand van een helikopter. Dit wil zeggen dat er een systeem ontwikkeld en geïmplementeerd zal moeten worden zodat de drone weet wanneer het uit de weg moet gaan voor de helikopter of dat een gecertifieerd piloot de drone uit de weg navigeert. De drone in de lucht houden heeft dus duidelijk heel wat nadelen. Als alle stroom gebruikt kan worden voor signalisatie, dan is er veel minder risico dat de drone moet terugkeren vooraleer de drenkeling gered werd door reddingsdiensten. Een nadeel is echter wel dat het voor de drone veel moeilijker is om de drenkeling in de gaten te houden. Er moet ook gezorgd worden dat de drone geen obstakel vormt voor de reddingsactie indien men boten gebruikt. Een goede oplossing zou kunnen zijn dat de drone uitgerust is met een airbagachtig systeem zodat het een reddingsboei kan vormen. De drone kan op een veilige afstand landen en de drenkeling kan er dan naartoe zwemmen. Dit maakt het ook gemakkelijker om via een microfoon te communiceren met mensen op het schip. 

\paragraph{Kalmte van de drenkeling bewaren}

Paniek is iets heel gevaarlijk in elke mogelijke situatie die je kan bedenken. Het is al vaak gebleken dat mensen, die in paniek slaan, het vermogen om rationeel na te denken verliezen. Dit feit leidt dikwijls tot ongelukkige en soms zelfs dodelijke ongevallen. Op zee kan paniek echter nog andere gevolgen teweeg brengen. Indien een drenkeling in paniek begint te slaan, zal hij of zij meer energie gaan verbruiken door hyperventilatie, overbodige bewegingen en eventueel door huilen of roepen om hulp. We moeten ervoor zorgen dat de drenkeling deze extra energie spaart zodat hij of zij niet bezwijkt door vermoeidheid.