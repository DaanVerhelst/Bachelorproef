%%=============================================================================
%% Inleiding
%%=============================================================================

\chapter{\IfLanguageName{dutch}{Inleiding}{Introduction}}
\label{ch:inleiding}

\section{\IfLanguageName{dutch}{Probleemstelling}{Problem Statement}}
\label{sec:probleemstelling}

Op jaarbasis sterven er gemiddeld 320.000 mensen de verdrinkingsdood. Het is echter wel zo dat ongeveer 75\% van deze sterfgevallen, ramp zoals overstromingen als oorzaak hebben, maar dan nog is het zo dat ongeveer 80.000 mensen buiten een ramp overlijden aan verdrinking. \autocite{WorldHealthOrganisation} Eén van de hoofdoorzaken van de verdrinkingsdood is uiteraard het niet op tijd terug vinden van de drenkelingen. Daarom zou het gebruik van een drone een groot verschil kunnen maken. Drones zijn sneller, efficiënter in objectherkenning dankzij de computer die het bestuurt en ze kunnen van bovenaf veel meer zien dan de reddingsdiensten zelf. Daarom kan de aanwezigheid van een drone op elk schip een goede oplossing zijn om mensen sneller op te sporen en zo het aantal sterfgevallen door verdrinking nog meer te beperken.

\section{\IfLanguageName{dutch}{Onderzoeksvragen}{Research question}}
\label{sec:onderzoeksvraag}

Welke hardware is het beste om te gebruiken bij reddingsacties op zee?

\begin{itemize}
	\item Welke camera's dient men te gebruiken?
	\item Welke drone(s) dient men te gebruiken?
	\item Welke extra opties kunnen de efficiëntie van reddingsacties nog verhogen?
\end{itemize}

Hoe kunnen we de drone het beste besturen?

\begin{itemize}
	\item Is een automatisch pilootsysteem voldoende? Moeten we combineren met handmatige besturing?
\end{itemize}

Wat is de beste manier om drones te implementeren in de bestaande situatie van de zeevaartsector?

\section{\IfLanguageName{dutch}{Onderzoeksdoelstelling}{Research objective}}
\label{sec:onderzoeksdoelstelling}

Op het einde van dit onderzoek moeten we in staat zijn om een aanbeveling te geven voor zowel de hardware als de software die gebruikt kunnen worden voor het inzetten van drones bij reddingsacties. Ook moeten we concreet kunnen aantonen hoe het gebruik van drones binnen het huidige systeem van de zeevaartsector kan verwerkt worden.

\section{\IfLanguageName{dutch}{Opzet van deze bachelorproef}{Structure of this bachelor thesis}}
\label{sec:opzet-bachelorproef}

% Het is gebruikelijk aan het einde van de inleiding een overzicht te
% geven van de opbouw van de rest van de tekst. Deze sectie bevat al een aanzet
% die je kan aanvullen/aanpassen in functie van je eigen tekst.

De rest van deze bachelorproef is als volgt opgebouwd:

In Hoofdstuk~\ref{ch:stand-van-zaken} wordt een overzicht gegeven van de stand van zaken binnen het onderzoeksdomein, op basis van een literatuurstudie.

In Hoofdstuk~\ref{ch:implementatie} wordt de een mogelijkheid voor de implementatie van dit concept toegelicht.

In Hoofdstuk~\ref{ch:proofofconcept} wordt beschreven hoe we een deel van ons concept uitgetest hebben.

% TODO: Vul hier aan voor je eigen hoofstukken, één of twee zinnen per hoofdstuk

In Hoofdstuk~\ref{ch:conclusie}, tenslotte, wordt de conclusie gegeven en een antwoord geformuleerd op de onderzoeksvragen. Daarbij wordt ook een aanzet gegeven voor toekomstig onderzoek binnen dit domein.