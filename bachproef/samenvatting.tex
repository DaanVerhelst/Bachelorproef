%%=============================================================================
%% Samenvatting
%%=============================================================================
% TODO: De "abstract" of samenvatting is een kernachtige (~ 1 blz. voor een
% thesis) synthese van het document.
%
% Deze aspecten moeten zeker aan bod komen:
% - Context: waarom is dit werk belangrijk?
% - Nood: waarom moest dit onderzocht worden?
% - Taak: wat heb je precies gedaan?
% - Object: wat staat in dit document geschreven?
% - Resultaat: wat was het resultaat?
% - Conclusie: wat is/zijn de belangrijkste conclusie(s)?
% - Perspectief: blijven er nog vragen open die in de toekomst nog kunnen
%    onderzocht worden? Wat is een mogelijk vervolg voor jouw onderzoek?
%
% LET OP! Een samenvatting is GEEN voorwoord!
%%---------- Nederlandse samenvatting -----------------------------------------
%
% TODO: Als je je bachelorproef in het Engels schrijft, moet je eerst een
% Nederlandse samenvatting invoegen. Haal daarvoor onderstaande code uit
% commentaar.
% Wie zijn bachelorproef in het Nederlands schrijft, kan dit negeren, de inhoud
% wordt niet in het document ingevoegd.
%\IfLanguageName{english}{%
%\selectlanguage{dutch}
%\chapter*{Samenvatting}
%\lipsum[1-4]
%\selectlanguage{english}
%}{}
%%---------- Samenvatting -----------------------------------------------------
%De samenvatting in de hoofdtaal van het document

\chapter*{\IfLanguageName{dutch}{Samenvatting}{Abstract}}

Jaarlijks sterven er wereldwijd gemiddeld 320.000 mensen de verdrinkingsdood. Ongeveer 75\% van die 320.000 sterfgevallen komen voort uit overspoeling rampen. Ook vissers hebben een verhoogd risico om te sterven door verdrinking. Daarom is het van cruciaal belang dat er een manier is die ten allen tijde beschikbaar is en helpt bij het zo spoedig mogelijk opsporen van drenkelingen. Een oplossing zou kunnen zijn dat we een drone op elk schip plaatsen die dan meteen ingezet kan worden voor het zoeken en volgen van de drenkeling alsook het doorsturen van zijn locatie. In deze paper bespreken we de verschillende camera- en dronetechnologieën die beschikbaar zijn om zo de ideale hardware uit te kiezen. Ook zullen we gaan kijken naar de software die nodig is om dit concept te kunnen realiseren. Daarna komen de belangrijkste factoren waarmee rekening gehouden moet worden voor de software.
Tenslotte bestuderen we het bestaande systeem waarin we ons concept willen implementeren om zo een zo goed mogelijk plan op te stellen. Daarvoor gaan we kijken naar procedures en wetgevingen die bedrijven en de reddingsdiensten moeten/kunnen volgen. Om dit concept effectief te gaan realiseren zullen er op Europees en/of wereldwijd vlak een aantal wetten of overeenkomsten gerealiseerd moeten worden omtrent het gebruik van drones voor reddingsoperaties. Indien deze wetgevingen en/of overeenkomsten tot stand gebracht kunnen worden, is er niets dat het implementeren van dit concept kan belemmeren. De nodige technologieën zijn beschikbaar en blijven nog steeds evolueren om nog betere prestaties te kunnen leveren. Het is dus zeker mogelijk dat de onderdelen die in deze paper besproken werden, vervangen kunnen worden door betere onderdelen in de toekomst. Er zijn ook extra zaken die we in de toekomst zouden kunnen onderzoeken zijn o.a. de werking van het softwarepakket dat de communicatie tussen het moederschip en de reddingsdiensten zal verzorgen. Ook zouden we in de toekomst onderzoek kunnen doen naar een systeem waarbij schepen, die in de buurt van het incident aanwezig zijn en over een drone beschikken, ook hun drone kunnen inzetten om nog efficiënter tewerk te kunnen gaan.
