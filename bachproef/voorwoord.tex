%%=============================================================================
%% Voorwoord
%%=============================================================================

\chapter*{\IfLanguageName{dutch}{Woord vooraf}{Preface}}
\label{ch:voorwoord}

%% TODO:
%% Het voorwoord is het enige deel van de bachelorproef waar je vanuit je
%% eigen standpunt (``ik-vorm'') mag schrijven. Je kan hier bv. motiveren
%% waarom jij het onderwerp wil bespreken.
%% Vergeet ook niet te bedanken wie je geholpen/gesteund/... heeft
Verdrinken is een verschrikkelijke manier om uit het leven te stappen. Ook is het iets dat mogelijks vermeden kan worden indien voldoende voorzorgsmaatregelen genomen worden. Helaas zijn er, ondanks alle huidige technologische mogelijkheden, nog steeds veel incidenten waarbij 1 of meerdere mensen overleiden aan de verdrinkingsdood. Daarom denk ik dat het van cruciaal belang is om zoveel mogelijk extra voorzorgsmaatregelen te zoeken en te implementeren. Een drone op het schip aanwezig hebben is daar een goed voorbeeld van omdat dit de zoektocht naar vermiste personen substantieel kan versnellen en zo de overlevingskansen van de drenkeling verhogen. Ook biedt een drone verscheidene extra opties die ook weer de overlevingskans kunnen verhogen. Daarom vindt ik dit een heel interessant onderwerp met veel use cases om het in toe te passen. 

Graag zou ik een aantal mensen willen bedanken voor alle hulp die zij verschaft hebben tijdens het uitschrijven van dit concept. Deze mensen hebben cruciale informatie bijgedragen over verscheidene onderdelen van het concept.

Eerst en vooral wil ik Mr. Jens Buysse bedanken. Hij heeft me gedurende het uitschrijven van deze paper, als promotor, ondersteund door het delen van een aantal van zijn ideeën rond het onderwerp alsook door het geven van feedback op verscheidene elementen zoals opmaak, spelling en veel meer.

Ten tweede wil ik Mevr. An Beun bedanken. Zij is een werknemer bij het IKWV of de Intercommunale Kustreddingsdienst West-Vlaanderen. Ook al was zij niet in staat om de nodige informatie te verschaffen, was Mevr. Beun wel bereid om zo goed mogelijk te kunnen helpen. Zij heeft mij doorverwezen naar het MRCC waar ik wel de nodige informatie kon vergaren.

Mr. Dries Boodts is een werknemer bij het MRCC of het Maritiem Reddings- en coördinatiecentrum en heeft heel wat bijgedragen aan het uiteindelijke resultaat van deze paper. Hij had veel te vertellen rond de wetgevingen met betrekking tot het gebruik van drones op zee alsook de procedures die zowel bedrijven als reddingsdiensten moeten/kunnen volgen. 

Mr. Pieter Verhelst was mijn copromotor voor deze paper. Hij ging mij bijstaan gedurende de proof-of-concept fase van de bachelorproef maar helaas stak het coronavirus daar een stokje voor. Toch wil ik hem bedanken omdat hij ertoe bereid was in de eerste plaats.  


