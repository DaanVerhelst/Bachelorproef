%%=============================================================================
%% Voorwoord
%%=============================================================================

\chapter*{\IfLanguageName{dutch}{Woord vooraf}{Preface}}
\label{ch:voorwoord}

%% TODO:
%% Het voorwoord is het enige deel van de bachelorproef waar je vanuit je
%% eigen standpunt (``ik-vorm'') mag schrijven. Je kan hier bv. motiveren
%% waarom jij het onderwerp wil bespreken.
%% Vergeet ook niet te bedanken wie je geholpen/gesteund/... heeft
Waarom koos ik voor dit onderwerp? Drones zijn de dag van vandaag niet meer uit het straatbeeld weg te denken. Hoewel je ze niet overal waar je kijkt ziet rondvliegen, worden ze al jaren op verschillende vlakken ingezet.

Zoals vele nieuwe technologische ontwikkelingen vindt ook deze zijn oorsprong in een militaire toepassing. Je hebt waarschijnlijk al een aantal beelden gezien van Amerikaanse "precisie-bombardementen" en in films worden dit soort beelden ook heel graag gebruikt.

Het is maar al te vaak zo dat dergelijke uitvindingen pas daarna hun weg vinden naar het commerciële circuit. Denken we hierbij maar aan luchtfotografie tot koerierdiensten zoals amazon. Wat echter nooit uit het oog mag verloren worden is het humane aspect. Hoe kunnen we drones gebruiken in het voordeel van de mens. Ook hier zijn gelukkig al tal van toepassingsgebieden zoals tijdens acties van politie en brandweer waar het van belang is om een algemeen overzichtsbeeld te krijgen van de situatie.

Door met verschillende mensen te praten over het onderwerp van mijn bachelorproef, vernam ik ook dat de eerste stappen al gezet zijn op het vlak van inzetten van drones tijdens reddingsoperaties rond en op zee. Zowel voor drenkelingen als bootvluchtelingen is het van levensbelang dat zij zo snel mogelijk gevonden en gered worden. 

Tijdens deze gesprekken kwam ook naar voor dat, naast de kinderziektes van de technologie, rond deze problematiek nog tal van keuzes moeten gemaakt worden betreffende hard- en software, alsook beslissingen omtrent éénduidige wetgeving eens men zich in internationale wateren bevindt. 
Dit leek mij dan ook een heel interessant deelaspect van de toepassing van drones, de moeite waard om verder uit te diepen. 

Ten eerste zijn er tal van vragen betreffende de drones die beantwoord moeten worden. Hoe kunnen ze ingezet worden worden voor het redden van mensenlevens op zee? Welke drones zijn er vandaag op de markt beschikbaar? Hoe zijn ze gemaakt en wat kosten ze? Welke zijn het meest geschikt in deze specifieke toepassing? Met welke opties kunnen ze verder uitgerust worden teneinde de efficiëntie te verhogen? Welke software is al ontwikkeld en waar kan deze nog verbeterd worden? 

Ten tweede stelt zich de vraag welke instanties zijn er nu al actief op het vlak van reddingsoperaties en hoe gaan zij te werk. In welke mate kunnen drones een meerwaarde betekenen in dergelijke situaties? Hoe kan er samengewerkt worden tussen deze instanties? En tot slot, minstens even belangrijk, welke wetgeving bestaat er en in welke mate kan hiervoor een internationaal draagvlak gecreëerd worden?

Vooraleer hierop verder in te gaan, wil ik een aantal mensen bedankten. Zij hebben cruciale informatie bijgedragen over verscheidene onderdelen van het concept of mij geholpen met de proof-of-concept.

Allereerst zou ik graag Mr. Dirk Vanbeylen willen bedanken. Hij heeft mij geholpen met het opstellen van het proof-of-concept door mij een drone ter beschikking te stellen alsook mij te helpen met de besturing ervan. Ook gaf hij mij allerhande tips en advies hieromtrent.

Ten tweede wil ik Mevr. An Beun bedanken. Zij is een werknemer bij het IKWV, kort voor de Intercommunale Kustreddingsdienst West-Vlaanderen. Al was zij niet in staat om zelf de nodige informatie te verschaffen, heeft ze me toch zo goed mogelijk geholpen. Zij heeft mij doorverwezen naar het MRCC waar ik wel de nodige informatie kon verzamelen.

Mr. Dries Boodts is werkzaam bij het MRCC, kort voor het Maritiem Reddings- en coördinatiecentrum, en heeft heel wat bijgedragen aan het uiteindelijke resultaat van deze paper. In het bijzonder kon hij mij veel vertellen rond de wetgevingen van toepassing op het gebruik van drones op zee alsook de procedures die zowel bedrijven als reddingsdiensten moeten/kunnen volgen. 

Mr. Pieter Verhelst was mijn copromotor voor deze paper. Hij zou me bijstaan gedurende de proof-of-concept fase van de bachelorproef maar helaas stak het coronavirus daar een stokje voor. Toch wil ik hem bedanken omdat hij ertoe bereid was in de eerste plaats.

Tenslotte wil ik Mr. Jens Buysse bedanken. Hij heeft me gedurende het uitschrijven van deze paper, als promotor, ondersteund door het delen van een aantal van zijn ideeën rond het onderwerp alsook door het geven van feedback op verscheidene elementen zoals opmaak, spelling en veel meer.


