%%=============================================================================
%% Voorwoord
%%=============================================================================

\chapter*{\IfLanguageName{dutch}{Woord vooraf}{Preface}}
\label{ch:voorwoord}

%% TODO:
%% Het voorwoord is het enige deel van de bachelorproef waar je vanuit je
%% eigen standpunt (``ik-vorm'') mag schrijven. Je kan hier bv. motiveren
%% waarom jij het onderwerp wil bespreken.
%% Vergeet ook niet te bedanken wie je geholpen/gesteund/... heeft
Verdrinken is een verschrikkelijke manier om aan je einde te komen. Ook is het iets dat mogelijks vermeden kan worden indien voldoende voorzorgsmaatregelen genomen worden. Helaas zijn er, ondanks alle huidige technologische ontwikkelingen, nog steeds veel incidenten waarbij 1 of meerdere mensen de verdrinkingsdood sterven. Daarom denk ik dat het van cruciaal belang is om zoveel mogelijk extra voorzorgsmaatregelen te zoeken en te implementeren. Een drone op het schip aanwezig hebben is daar een goed voorbeeld van omdat dit de zoektocht naar vermiste personen substantieel kan versnellen en zo de overlevingskansen van de drenkeling verhogen. Ook biedt een drone verscheidene extra opties die de overlevingskans verhogen. Daarom vind ik dit een heel interessant onderwerp met veel use cases om het in toe te passen. 

Graag zou ik een aantal mensen willen bedanken voor alle hulp die ze me geboden hebben tijdens het uitschrijven van dit concept. Zij hebben cruciale informatie bijgedragen over verscheidene onderdelen van het concept.

Allereerst zou ik graag Mr. Dirk Vanbeylen willen bedanken. Hij heeft mij geholpen met het opstellen van de proof-of-concept door mij een drone ter beschikking te stellen alsook mij te helpen met de besturing ervan. Ook gaf hij mij allerhande tips en advies hieromtrent.

Ten tweede wil ik Mevr. An Beun bedanken. Zij is een werknemer bij het IKWV, kort voor de Intercommunale Kustreddingsdienst West-Vlaanderen. Ook al was zij niet in staat om zelf de nodige informatie te verschaffen, toch heeft ze me zo goed mogelijk geholpen. Zij heeft mij doorverwezen naar het MRCC waar ik wel de nodige informatie kon verzamelen.

Mr. Dries Boodts is werkzaam bij het MRCC of het Maritiem Reddings- en coördinatiecentrum en heeft heel wat bijgedragen aan het uiteindelijke resultaat van deze paper. In het bijzonder kon hij mij veel vertellen rond de wetgevingen van toepassing op het gebruik van drones op zee alsook de procedures die zowel bedrijven als reddingsdiensten moeten/kunnen volgen. 

Mr. Pieter Verhelst was mijn copromotor voor deze paper. Hij zou me bijstaan gedurende de proof-of-concept fase van de bachelorproef maar helaas stak het coronavirus daar een stokje voor. Toch wil ik hem bedanken omdat hij ertoe bereid was in de eerste plaats.

Tenslotte wil ik Mr. Jens Buysse bedanken. Hij heeft me gedurende het uitschrijven van deze paper, als promotor, ondersteund door het delen van een aantal van zijn ideeën rond het onderwerp alsook door het geven van feedback op verscheidene elementen zoals opmaak, spelling en veel meer.


