%%=============================================================================
%% Conclusie
%%=============================================================================

\chapter{Conclusie}
\label{ch:conclusie}

% TODO: Trek een duidelijke conclusie, in de vorm van een antwoord op de
% onderzoeksvra(a)g(en). Wat was jouw bijdrage aan het onderzoeksdomein en
% hoe biedt dit meerwaarde aan het vakgebied/doelgroep? 
% Reflecteer kritisch over het resultaat. In Engelse teksten wordt deze sectie
% ``Discussion'' genoemd. Had je deze uitkomst verwacht? Zijn er zaken die nog
% niet duidelijk zijn?
% Heeft het onderzoek geleid tot nieuwe vragen die uitnodigen tot verder 
%onderzoek?

We hebben de hardware, de software en de implementatie van een on-board drone besproken en kunnen besluiten dat dit de beste manier van aanpak is. 

Van cruciaal belang is het feit dat op Europees en/of globaal vlak een éénduidige wetgeving tot stand komt zodat dit concept kan geïmplementeerd worden. 

Eens dat er is, kunnen we de duurdere drone gebruiken die standaard uitgerust is met een warmtecamera en een infrarood camera alsook een systeem dat ons in staat stelt om zaken te transporteren. 

Vervolgens zullen we een groot volume voorbeeldfoto's moeten maken om deze onder te verdelen tussen training- en testdataset. Pas daarna kunnen we het klassificatie-algoritme trainen en aan de hand van de testdataset, de accuraatheid gaan testen. Eens de accuraatheid van het algoritme op punt staat, kunnen we het algoritme in gebruik gaan nemen. 

Verder kunnen we de drone verder gaan uitrusten met een extra microfoon, luidspreker, een reddingsboei of module die de drone omvormt tot een reddingsband en het belangrijkste onderdeel, een lokalisatiezender zodat de locatie doorgegeven kan worden aan zowel de bemanning van het schip als de reddingsdiensten. 

Tot slot kunnen we nog het softwarepakket gaan maken dat de drone in staat stelt om de kustwacht, verantwoordelijk voor de wateren waarin het schip zich bevindt, te verwittigen van het incident alsook andere gegevens zoals de locatie van het incident door te geven. Ook zal het softwarepakket instaan voor de verdere communicatie tussen de crew van het schip en diezelfde kustwacht.

De stappen die men moet ondernemen in het geval van een man-over-boord situatie zijn de volgende:

De eerste stap is het lanceren van de drone zodat deze een signaal geeft aan de respectievelijke reddingsdiensten. Vervolgens wordt er gewacht op het contact met de reddingsdiensten. Deze hebben ondertussen de locatie reeds doorgegeven aan de reddingsoperatoren. Verder advies kan gegeven worden door de reddingsdiensten aan de crew. Tenslotte kan de drone, éénmaal de drenkeling gevonden is, de locatie doorgeven zodat de reddingsdiensten (of de crew indien de reddingsdiensten nog niet aangekomen zijn) kunnen overgaan tot de effectieve reddingsoperatie.

Na het uitvoeren van de proof-of-concept, kunnen we nu met zekerheid zeggen dat de technologieën, die op dit moment op de markt beschikbaar zijn, zeker volstaan om dit concept volledig uit te werken. Het is zelfs mogelijk om de drone alles op een volledig autonome manier te laten uitvoeren. Zeker met de duurdere drones die veel flexibeler geprogrammeerd kunnen worden dankzij de extra sensoren, betere vliegperformantie en andere optionele uitbreidingen moet dit concept zonder twijfel gerealiseerd kunnen worden. Het is echter wel zo dat dit een omvangrijk project zal zijn aangezien, zoals eerder al vermeld, er heel wat invalshoeken zijn waarmee rekening moet gehouden worden. Voor dergelijk project kunnen de kosten hoog oplopen en zal de tijd die dit in beslag kan nemen aanzienlijk zijn. Kosten noch moeite mogen echter gespaard blijven wanneer we het hebben over het redden van een mensenleven.

 

