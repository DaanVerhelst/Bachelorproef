%%=============================================================================
%% Conclusie
%%=============================================================================

\chapter{Conclusie}
\label{ch:conclusie}

% TODO: Trek een duidelijke conclusie, in de vorm van een antwoord op de
% onderzoeksvra(a)g(en). Wat was jouw bijdrage aan het onderzoeksdomein en
% hoe biedt dit meerwaarde aan het vakgebied/doelgroep? 
% Reflecteer kritisch over het resultaat. In Engelse teksten wordt deze sectie
% ``Discussion'' genoemd. Had je deze uitkomst verwacht? Zijn er zaken die nog
% niet duidelijk zijn?
% Heeft het onderzoek geleid tot nieuwe vragen die uitnodigen tot verder 
%onderzoek?

We hebben de hardware, de software en de implementatie van een on-ship drone besproken en hebben geconcludeerd dat dit de beste manier van aanpak is. 

Eerst en vooral zal er een overeenkomst op Europees of wereldwijd niveau gemaakt moeten worden met verschillende landen (bij voorkeur zo veel mogelijk) om het concept te mogen en kunnen implementeren. 

Eens dat er is, kunnen we de duurdere drone gebruiken die standaard uitgerust is met een warmtecamera en een infrarood camera alsook een systeem dat ons in staat stelt om zaken te transporteren. 

Vervolgens zullen we een groot volume voorbeeldfoto's moeten maken om deze onder te verdelen tussen training- en testdataset. Dan kunnen we het classificatiealgoritme trainen en aan de hand van de testdataset, de accuraatheid gaan testen. Eens de accuraatheid van het algoritme naar wens is kunnen we het algoritme in gebruik gaan nemen. 

Verder moeten we de drone verder gaan uitrusten met een extra microfoon, luidspreker, een reddingsband of module dat de drone omvormt tot een reddingsband en het belangrijkste onderdeel, een lokalisatiezender zodat de locatie doorgegeven kan worden aan zowel de bemanning van het schip als de reddingsdiensten. 

Nu moeten we nog het softwarepakket gaan maken dat de drone in staat stelt om de kustwacht, dat instaat voor de wateren waarin het schip zich bevindt, te verwittigen van het incident alsook andere gegevens zoals de locatie van het incident door te geven. Ook zal het softwarepakket instaan voor de verdere communicatie tussen de crew van het schip en diezelfde kustwacht.

De stappen die men moet ondernemen in het geval van een man-over-boord incident is als volgt.

De eerste stap is het lanceren van de drone zodat deze een signaal geeft aan de respectievelijke reddingsdiensten. Vervolgens wordt er gewacht op het contact van de reddingsdiensten. Deze hebben ondertussen de locatie reeds doorgegeven aan de reddingsoperatoren. Verder advies kan gegeven worden door de reddingsdiensten aan de crew. Eens de drone de drenkeling opspoort geeft hij de locatie door en kunnen de reddingsdiensten (of de crew indien de reddingsdiensten nog niet aangekomen zijn) overgaan tot de redding van de drenkeling.

 

