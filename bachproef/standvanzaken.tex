\chapter{\IfLanguageName{dutch}{Stand van zaken}{State of the art}}
\label{ch:stand-van-zaken}

% Tip: Begin elk hoofdstuk met een paragraaf inleiding die beschrijft hoe
% dit hoofdstuk past binnen het geheel van de bachelorproef. Geef in het
% bijzonder aan wat de link is met het vorige en volgende hoofdstuk.

% Pas na deze inleidende paragraaf komt de eerste sectiehoofding.

\section{Hardware}

\paragraph{Inleiding}
In dit deel van de paper zullen we de noodzakelijke en optionele hardware bespreken die gebruikt kunnen/moeten worden voor het optimaliseren van de efficiëntie van drones bij het detecteren van drenkelingen op diepzee. We nemen even de tijd om de verschillende beschikbare types van camera's te bestuderen. We zullen proberen om de meest geschikte camera te vinden voor het assisteren bij reddingsoperaties. Bij het vinden van deze optimale camera moet uiteraard rekening gehouden worden met verscheidene factoren zoals kledij, snelheid van herkenning van drenkelingen, etc. Ook zullen we een vergelijking maken tussen verschillende drones die momenteel beschikbaar zijn. Factoren zoals snelheid, stabiliteit, robuustheid, kostprijs, etc zullen hier van groot belang zijn. We zullen ook de voor en nadelen van zowel één als meerdere drones inzetten bespreken. Ten slotte bespreken we nog enkele optionele onderdelen die eventueel ook gebruikt kunnen worden voor het optimaliseren van de overlevingskans van de drenkeling. Hier denken we aan een mogelijkheid tot communicatie tussen boot en drenkeling, het brengen van een reddingsboei, etc.

\subsection{Optimale optische technologie voor het detecteren van drenkelingen}

\paragraph{Normale camera}

\lipsum[1-2]

\paragraph{IR-Camera}

\lipsum[1-2]

\paragraph{Warmtecamera}

\lipsum[1-2]

\paragraph{Nachtzichtcamera}

\lipsum[1-2]

\paragraph{Extra factoren}

\lipsum[1-2]

\subsection{Optimale drone voor reddingsacties op zee}

\paragraph{Een versus meerdere drones}

\lipsum[1-2]

\paragraph{Fysieke eigenschappen}

\lipsum[1-2]

\paragraph{Kostprijs}

\lipsum[1-2]

\subsection{Extra opties}

\paragraph{Landingsapparatuur voor waterlandingen}

\lipsum[1-2]

\paragraph{Robotische arm voor het transporteren van items}

\lipsum[1-2]

\paragraph{Luidsprekers en een microfoon voor communicatie}

\lipsum[1-2]

\paragraph{Vuurfakkel voor signalisatie}

\lipsum[1-2]

\section{Software}

\paragraph{Inleiding}
In vorige sectie hebben we de noodzakelijke en optionele hardware, zoals drones en camera's, besproken. Uiteraard is hardware op zichzelf niet voldoende. Deze hardware moet door een of andere software bestuurd en ondersteund worden en dat is precies wat we in deze sectie gaan bespreken. We zullen onderzoek doen naar het al dan niet handmatig of ge-automatiseerd besturen van de drone. Een combinatie is uiteraard ook een optie. Daarna zullen de voordelen en nadelen van verscheidene herkenningsalgorithmen gaan vastleggen om zo de meest optimale algorithmen eruit te kunnen kiezen.

\subsection{Besturing van de drone}

\paragraph{Handmatig}

\lipsum[1-2]

\paragraph{Automatisch}

\lipsum[1-2]

\paragraph{Combinatie}

\lipsum[1-2]

\subsection{Herkenningssoftware}

\paragraph{Voordelen}

\lipsum[1-2]

\paragraph{Nadelen}

\lipsum[1-2]

\section{Implementatie}

\paragraph{Inleiding}

Nu dat we voldoende voorkennis hebben omtrent zowel hardware als software alsook de meest efficiënte hardware en software bepaald hebben voor deze toepassing (nl. reddingsacties op diepzee) kunnen we verder gaan naar de implementatiefase van dit concept. Voordat we een goed voorstel kunnen geven voor de implementatie van drones bij reddingsacties moeten we uiteraard een goed beeld hebben van de huidige situatie binnen de zeevaartsector. Daarom zullen we ons in deze sectie eerst gaan verdiepen in de bestaande procedures die bedrijven en reddingsdiensten moeten volgen, alsook de wetten rond het gebruik van drones. Daarna zullen we, met alle vooraf vergaarde kennis, de effectieve acties en implementatie van de drone gaan vastleggen. We bespreken dan alle stappen van opstijgen en landen tot het vastleggen van acties die de drone zou ondernemen in het geval van het detecteren van een drenkeling. 

\subsection{Bestaande procedures en verwerking}

\paragraph{Wetten rond dronegebruik}

\lipsum[1-2]

\paragraph{Procedures van Reddingsdiensten}

\lipsum[1-2]

\paragraph{Procedures van bedrijven in de scheepvaartsector}

\lipsum[1-2]

\paragraph{Verwerking van drones in bestaand systeem}

\lipsum[1-2]

\subsection{Opstijgen en landen op zee}

\paragraph{Vertrek}

\lipsum[1-2]

\paragraph{Recovery}

\lipsum[1-2]

\subsection{Acties bij detectie van een drenkeling}

\paragraph{Signalisatie van locatie}

\lipsum[1-2]

\paragraph{Landen versus vliegen}

\lipsum[1-2]

\paragraph{Kalmte van de drenkeling bewaken}

\lipsum[1-2]
