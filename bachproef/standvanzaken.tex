\chapter{\IfLanguageName{dutch}{Stand van zaken}{State of the art}}
\label{ch:stand-van-zaken}

% Tip: Begin elk hoofdstuk met een paragraaf inleiding die beschrijft hoe
% dit hoofdstuk past binnen het geheel van de bachelorproef. Geef in het
% bijzonder aan wat de link is met het vorige en volgende hoofdstuk.

% Pas na deze inleidende paragraaf komt de eerste sectiehoofding.

\section{Hardware}

\subsection{Inleiding}

In dit deel zullen we de noodzakelijke en optionele hardware bespreken die gebruikt kunnen/moeten worden voor het optimaliseren van de efficiëntie van drones bij het detecteren van drenkelingen op \\(diep\\)zee. We nemen even de tijd om de verschillende beschikbare types van camera's te bestuderen. We zullen proberen om de meest geschikte camera('s) te vinden voor het assisteren bij reddingsoperaties. Bij het vinden van deze optimale camera('s) moet er uiteraard rekening gehouden worden met verscheidene factoren zoals de kledij van de drenkeling (kleur, reflectiviteit), de snelheid waarmee drenkelingen gedetecteerd kunnen worden, etc. Ook zullen we een vergelijking maken tussen verschillende drones die momenteel beschikbaar zijn. Factoren zoals snelheid, stabiliteit, robuustheid, kostprijs, etc zullen hier van groot belang zijn. Ook gaan we de voor en nadelen bespreken van zowel één als meerdere drones inzetten. Ten slotte bespreken we nog enkele optionele onderdelen die eventueel ook gebruikt kunnen worden voor het optimaliseren van de overlevingskans van de drenkeling. Hier denken we aan een mogelijkheid tot communicatie tussen de boot en de drenkeling, het brengen van een reddingsboei naar de drenkeling, etc.

\subsection{Optimale optische technologie voor het detecteren van drenkelingen}

\paragraph{Standaard camera}

De standaard camera, die we gebruiken voor het maken van mooie foto's, is ons allemaal bekend. Elke smartphone heeft tegenwoordig camera's die prachtige en gedetailleerde foto's kunnen maken en de ontwikkeling van deze technologie is nog steeds niet aan het vertragen. Maar wat is zo'n camera nu eigenlijk? Hoe werkt het en hoe kunnen wij het gaan gebruiken voor de uitwerking van ons idee? (nl. het gebruik van camera's voor persoons detectie op zee)

Wat is een camera? Een camera is een toestel dat in staat is om een beeld dat zich voor de lens bevindt vast te leggen. Eens de persoon, die de foto wil trekken, op de knop drukt om het beeld vast te leggen, opent een shutter die tot dan het licht tegenhield. De lens zal dan het binnenkomende licht focussen zodat het op een fotografische film of digitale sensor terecht komt. Aan de hand van deze fotografische film of digitale sensor, kan dan een digitale representatie van de getrokken personen, het getrokken object of landschap samengesteld worden. Als we nu dit proces telkens weer herhalen en alle foto's op een snel tempo na elkaar tonen, krijg je een videofragment in plaats van één enkele foto. Om dit te kunnen doen moet men natuurlijk aan een zeer hoog tempo foto's kunnen trekken zodat verandering in de getrokken omgeving geleidelijk aan weergegeven kan worden en niet in sprongen. De snelheid waarmee de foto's genomen worden, wordt uitgedrukt in frames per seconde waarbij een frame een synoniem is voor een foto. Een veel voorkomend frame-rate bij videocamera's is 24 frames per seconde wat dus wil zeggen dat er per seconde 24 foto's genomen worden. Om deze video nu aan een correcte snelheid af te spelen moeten de afbeeldingen ook aan een snelheid van 24 frames per seconde weergegeven worden. De resolutie van een foto is ook een belangrijke factor bij fotografie. Deze wordt uitgedrukt in pixels en hangt af van het aantal fotodiodes op de digitale sensor. Deze fotodiodes zijn de componenten die de lichtinval opvangen en omzetten naar digitale data van de foto. Een pixel is een, meestal heel klein, stukje van de afbeelding die een bepaalde kleur aanneemt. \autocite{DigitalCameras}


Hoe kunnen wij zo'n camera nu gaan toepassen op de detectie van drenkelingen op diepzee? Zoals eerder vermeld werd, kunnen we aan een heel hoog tempo foto's nemen met een welbepaalde resolutie om zo een videofragment vast te leggen. Indien we nu elk van deze frames zouden laten analyseren aan de hand van een algoritme, die in staat is om mensen te herkennen/classificeren, zouden we op die manier aan persoonsherkenning kunnen doen. Het is van cruciaal belang dat we het frame-rate alsook de resolutie van de beelden niet te hoog maken zodat het algoritme niet overladen wordt bij het analyseren van alle frames. Een probleem met deze camera is de afhankelijkheid van voldoende lichtinval. In zeer donkere omstandigheden zoals de nacht op zee, zou dit dus geen goede oplossing zijn. 

\paragraph{IR-Camera}

Zoals we in de vorige paragraaf besproken hebben, is het zo dat normale camera's in donkere omstandigheden, niet voldoende zouden zijn voor het detecteren van drenkelingen op zee. Het is namelijk zo dat er gedurende de helft van de dag cyclus niet voldoende lichtinval zal zijn om duidelijke afbeeldingen te maken waaruit een algoritme dan een drenkeling kan herkennen. Daarom zijn er andere technologieën die op een zeer gelijkaardige manier beelden kunnen vastleggen maar toch met een ander resultaat. Een van deze alternatieve technologieën is de infrarood camera. 


Wat is nu het verschil tussen een gewone camera en een infrarood camera? Zoals eerder vermeld is het eigenlijk een zeer klein verschil tussen de twee camera's. Ook nu wordt er gewerkt met een lens, een shutter en sensoren. Het verschil is dat, bij een infrarood camera, een infrarood sensor array gebruikt wordt in plaats van een gewone digitale sensor. Deze sensoren zijn in staat om infrarood energie te detecteren en deze informatie om te zetten naar een afbeelding. Het menselijk lichaam geeft veel verschillende stralingen af maar infrarood licht is een van de meest aanwezige vormen van straling.


Om deze technologie nu te gaan toepassen op het detecteren van drenkelingen op zee, kunnen we zeer gelijkaardig te werk gaan. Een infrarood camera kan namelijk, op een gelijkaardige manier als de normale camera, heel snel foto's nemen. Op die manier kunnen we videomateriaal maken door ze opnieuw op eenzelfde framerate af te spelen. Ook kunnen we frame per frame gaan analyseren om na te gaan of een drenkeling op dat frame te zien is. Het voordeel aan infrarood camera's is dat, het menselijk lichaam, die veel infraroodstralingen uitstraalt, sterk oplicht op een frame of foto die genomen werd door een infrarood camera. Hierdoor is een infrarood camera een zeer goede optische technologie in donkere omgevingen. 

\paragraph{Warmtecamera}

Zoals vermeld in de bovenstaande paragraaf, zijn er betere alternatieven voor het detecteren van drenkelingen dan een gewone camera. De warmtecamera is hier nog een voorbeeld van. Deze camera is ook weer heel gelijkaardig aan de gewone camera met opnieuw als grootste verschil de soort straling die gedetecteerd wordt. Bij warmtecameras wordt er ook licht uit het infra-rode spectrum opgevangen maar deze keer in het lange infrarood bereik waar men bij infrarood camera's eerder het licht uit de near-infrarode regio gaat opvangen. Het licht binnen de near-infra-rode regio is net niet zichtbaar met het blote oog. Het grote nadeel van infrarood camera's is dat er toch nog een heel kleine hoeveelheid licht nodig is om de camera naar behoren te laten werken. Dit is waar warmtecamera's superieur zijn ten opzichte van infrarood camera's. Omdat warmtecamera's gebruik maken van licht uit het lange infrarood bereik, zijn deze camera's niet afhankelijk van ander licht. Dit type van camera registreert enkel de warmte die een mensenlichaam uitstraalt. Daarom is dit een betere optie ten opzichte van de andere camera's. Deze kan zowel 's nachts als overdag gebruikt worden. Een probleem is echter wel dat de lichaamstemperatuur van de drenkeling zal zakken in het koude zeewater. Een combinatie van infrarood met warmtecamera is dus aangeraden voor een maximale efficiëntie.

Deze camera kunnen we opnieuw op een gelijkaardige manier gebruiken. We laten foto's of frames nemen die we dan analyseren aan de hand van een algoritme. Op basis daarvan kunnen we bepalen of er een drenkeling in beeld is of niet. 

\paragraph{Conclusie}
Het is duidelijk dat de gewone camera niet voldoende zal zijn voor het efficiënt detecteren van drenkelingen. De beste opties zijn infrarood- en warmtecamera's. Aangezien elk van deze camera's ook nadelen hebben, is het beter om beide te combineren voor maximum efficiëntie.

\subsection{Optimale drone voor reddingsacties op zee}

Er zijn verschillende zaken die in rekening gebracht moeten worden wanneer we een drone kiezen. Zo moeten we bijvoorbeeld nagaan wat de kwaliteit is van de materialen die gebruikt werden bij de productie van de drone, de mate waarin het geprogrammeerd kan worden om autonoom op pad te gaan, de kostprijs en de beschikbaarheid van de drone en zijn bijbehorende vervangstukken, de reikwijdte van het toestel, de hefkracht van het toestel en hoelang de drone in de lucht kan blijven zonder opgeladen te moeten worden. Eerst zullen we een goedkopere consumer-grade drone bespreken. We kijken of het voordeliger is om goedkope toestellen te gebruiken en ze simpelweg te vervangen bij schade of dat we een duurdere drone nemen met vervangstukken.

\paragraph{Consumer-grade drone}

Het eerste toestel dat we gaan bekijken is een drone van het bedrijf Ryze Robotics. \autocite{CheapDrone} Het kost iets minder dan 110 euro wanneer we het aankopen via het officiële verkooppunt \autocite{CheapDroneOfficial}. 

We weten niet uit welk materiaal deze drone opgebouwd is wat wil zeggen dat we ook niet weten hoe resistent deze drone is tegen heftigere weersomstandigheden. De drone is niet programmeerbaar en er is geen mogelijkheid om een reddingsband naar de drenkeling te transporteren door een gebrek aan opties voor een robotische arm alsook een lage draagkracht van de drone. Ook herstelling van een drone is niet mogelijk door het gebrek aan vervangstukken dus moeten we bij schade een nieuwe drone aanschaffen. De reikwijdte van de drone is slechts 100 meter. Indien de drenkeling pas na een paar minuten als vermist opgegeven wordt, is het mogelijk dat de drone niet ver genoeg kan vliegen om de drenkeling te detecteren. De vliegtijd is slechts 13 minuten dus indien de drenkeling niet binnen die 13 minuten terug gevonden en gered wordt, verliest de drone zijn waarde aangezien de drone  in staat moet zijn om zijn locatie te versturen naar de reddingstroepen. De snelheid van 8 meter / seconde is ook aan de lage kant.
Over het algemeen kunnen we stellen dat deze drone niet geschikt is voor onze use case.

\paragraph{Drone uit de middelste prijsklasse}

De Mavic Pro, van het merk DJI, is een toestel met een prijs van 1000 euro. Dit is op zich niet goedkoop maar als je kijkt naar de prijzen die drones kunnen aannemen dan valt dit best nog mee. Dit is de drone waar we ook effectief onze proof-of-concept mee zullen uitvoeren. Deze drone kan tot 13km ver vliegen in optimale omstandigheden (weinig of geen wind). Het kan vliegen aan een maximale snelheid van 65 km per uur (opnieuw indien er weinig of geen wind is) en heeft een gewone camera die 4K beelden vast kan leggen. De drone kan tot 21 minuten vliegwaardig blijven. Daarna moet de batterij vervangen worden maar dit kan vrij snel en efficiënt gebeuren. Deze drone kan geautomatiseerd worden maar met een aantal beperkingen. We zullen dit later in meer diepte bespreken.

\paragraph{Een duurdere drone}

Nu gaan we een voorbeeld van een duurdere drone bekijken die specifiek ontworpen is voor het gebruik in search and rescue missies. Deze drone is heel wat duurder dan een consumer-grade drone maar zou heel wat voordelen hebben. \autocite{Industrial}

Deze drone is bijvoorbeeld gemaakt met materialen die water- en stofwerend zijn zodat het in verscheidene weersomstandigheden kan vliegen, deze drone kan geprogrammeerd worden om een bepaald patroon te volgen, er is standaard een systeem geïmplementeerd dat de drone in staat stelt om pakketjes van 1.4kg te transporteren en te laten vallen, de drone heeft een vliegbereik van 8000 meter, er is een warmtecamera en infrarood camera (die ook als normale camera kan functioneren) inbegrepen in het pakket (wat ideaal is gezien we deze twee camera's wensen te combineren), de drone kan 38 minuten vliegwaardig blijven vooraleer het opgeladen moet worden, de drone heeft een maximum snelheid van 82 kilometer per uur en er is een handige kist waar de drone en alle bijbehorende onderdelen in opgeborgen kunnen worden. Ook is er een monitor inbegrepen in het pakket.

Dit is duidelijk een betere optie voor deze use case.

\paragraph{Een versus meerdere drones}

Het snel vinden van drenkelingen op zee is cruciaal voor de overlevingskansen van deze persoon. Dit roept de volgende vraag op. Is één drone voldoende of hebben we er meerdere nodig om de efficiëntie van het zoeken te optimaliseren? Om dit te bepalen moeten we nagaan hoeveel oppervlakte één drone kan doorzoeken binnen een bepaald tijdsbestek. Neem nu dat de drone de maximum hoogte van 120 meter aanhoudt en dat de drone een oppervlakte van 6400 vierkante meter (80m x 80m) "ziet". Combineer dit nu met het feit dat de hierboven besproken drone een snelheid van 23m/s kan aanhouden. Hieruit kunnen we afleiden dat de drone, iedere 4 seconden, 64 vierkante meter kan inspecteren met als gevolg dat we in 1 minuut, 96 vierkante kilometer aan oppervlakte kunnen doorzoeken. We moeten er uiteraard rekening mee houden dat de drone deze snelheid niet continue aanhoudt maar zelfs al is het maar 80 viekante kilometer dan is dat nog steeds meer dan voldoende. Het is dus wel duidelijk dat één drone voldoende zal zijn voor het vinden van één drenkeling.

\subsection{Extra opties}

De drone zelf is op zich reeds een grote aanwinst bij reddingsoperaties maar het kan nog beter. Door de drone uit te rusten met een aantal extra opties, kunnen we de efficiëntie van een drone bij reddingsoperaties nog verhogen. Daarom zullen we in dit onderdeel een aantal optionele uitbreidingen bespreken waarmee de drone ook uitgerust kan worden.  

\paragraph{Luidsprekers en een microfoon voor communicatie}

Stel iemand valt overboord in het water. De drone wordt geactiveerd en vindt na een bepaalde tijd de drenkeling terug. Wat doet de drone terwijl de reddingsdiensten onderweg zijn? Een idee is om een luidspreker en microfoon in de drone te verwerken om communicatie op afstand mogelijk te maken. Zo kunnen we de drenkeling misschien kalmeren. Ook zouden we kalmerende muziek kunnen afspelen om opnieuw de kalmte te bewaren. 
De microfoon kan ook dienen als een geluidsdetector. Als de drenkeling luid roept naar de drone die hij/zij ziet passeren, dan zou dit ook als middel voor lokalisatie kunnen dienen. Dit is echter enkel mogelijk in rustige en stille omstandigheden. Het geluid van wind en water kunnen hierbij een storende factor zijn. 

\paragraph{Vuurfakkel voor signalisatie}

Een andere een optie zou kunnen zijn dat we aan de hand van het transportsysteem een vuurfakkel naar de drenkeling brengen zodat de reddingsoperatoren sneller zien waar de drenkeling zich bevindt. Dit zou tijd uitsparen in donkere omstandigheden. Hoewel de warmtecamera dan de exacte locatie uit het oog zal verliezen, door de grote hoeveelheid hitte die de vuurfakkel afgeeft, kan een schatting van de locatie nog steeds behouden worden en voor de exacte locatie dient uiteraard de vuurfakkel. 

\section{Software}

\paragraph{Inleiding}
In het vorige onderdeel hebben we de noodzakelijke en optionele hardware, zoals drones en camera's, besproken. Uiteraard is hardware op zichzelf niet voldoende. Deze hardware moet door een of ander software systeem bestuurd en ondersteund worden en dat is precies wat we in dit onderdeel gaan bespreken. We zullen onderzoek doen naar het al dan niet handmatig of geautomatiseerd besturen van de drone. Een combinatie is uiteraard ook een optie. Daarna zullen we het herkenningsalgoritme gaan bespreken zodat we een beter begrip krijgen van hoe het werkt en hoe we het kunnen toepassen op onze use case.

\subsection{Besturing van de drone}

\paragraph{Handmatig}

Het handmatig besturen van de drone heeft zijn voor- en nadelen. De voordelen zijn dat je meer controle hebt over de drone waardoor je gerichter kan werken. Als je weet waar de drenkeling ongeveer zou kunnen zijn zou je de drone zelf in die richting kunnen besturen om zo sneller de drenkeling te vinden. De nadelen zijn dat mensen niet altijd even accuraat tewerk kunnen gaan als we het vergelijken met een computer. Nog een nadeel van het handmatig besturen van de drone is dat men mensen zal moeten opleiden voor het besturen van die drone.

\paragraph{Autonoom}

De drone laten besturen door een computer systeem heeft ook zijn voor- en nadelen. De voordelen zijn dat de computer, op basis van coördinaten, een veel accurater rasterpatroon kan volgen en dus minder fouten maakt. Op die manier kan de drone, zo efficiënt als mogelijk is, grote oppervlakten doorzoeken zonder kleine stukjes zee-oppervlakte mis te lopen. Het grote nadeel hieraan is dat een computer systeem niet over intuïtie of buikgevoel beschikt. Het zal zijn voorgeprogrammeerd pad afleggen en niets meer of minder dan dat.  

\paragraph{Combinatie}

Gezien beide zowel voor- als nadelen heeft is het beter om de twee opties te combineren om op die manier zoveel mogelijk voordelen te behouden. We zouden bijvoorbeeld handmatig kunnen zoeken in de algemene richting waar men vermoed de drenkeling terug te vinden en na het vinden van de drenkeling kan de computer dan overnemen voor het volgen van de drenkeling tot hij of zij uiteindelijk gered wordt. Ook is het soms wettelijk verplicht om de drone door een gecertifieerde piloot te laten besturen. In dat geval kunnen we niet anders.

\subsection{Herkenningssoftware}

\paragraph{Wat is het?}

Hierboven hebben we reeds een aantal keer aangehaald dat we een algoritme zullen gebruiken voor het analyseren van de frames die we binnen krijgen. Dit algoritme heet een klassificatie neuraal netwerk of CNN en is een onderdeel van Aritficiële Intelligentie. Voordat we het algoritme in gebruik kunnen nemen, moeten we het gaan trainen aan de hand van een grote set voorbeeldafbeeldingen. Het algoritme zal proberen te "redeneren" tot welke klasse elke voorbeeld toebehoort en op basis van de fouten die het maakt zal een aanpassing in "de denkwijze" aangebracht worden. Deze "redeneringen" gebeuren aan de hand van een wiskundige functie en het "aanpassen van de denkwijze" is niets meer dan het veranderen van de variabelen van die functie. Nadat het algoritme voldoende getrained is, zou het dan in staat moeten zijn om afbeeldingen, die niet in de set voorbeelden zaten, juist te klassificeren. Het grootste nadeel van dit soort algoritme is dat we een groot aantal foto's ter beschikking moeten hebben van de zee, zowel met als zonder drenkeling.

\paragraph{Toepassen in onze usecase}

Zoals hierboven vermeld werd, moeten we dus een grote set voorbeeldfoto's gaan gebruiken om het algoritme op te leren in het herkennen van drenkelingen op een afbeelding. We hebben al gezegd dat we een warmtecamera en een infrarood camera gaan gebruiken dus we kunnen ervoor zorgen dat de personen op de afbeelding sterk verschillen in kleur ten opzichte van de rest van de afbeelding. De set voorbeeldfoto's zal evenveel afbeeldingen met als zonder drenkeling moeten bevatten en dan zullen we nog een kleinere set voorbeeldfoto's moeten voorzien om het algoritme te gaan testen op het accuraat klassificeren. Eens de training voltooid is, kan het algoritme ingezet worden om aan persoonsherkenning te doen.

\subsection{Hoe kan een drone geprogrammeerd worden?}

\paragraph{Drones van het merk DJI}

DJI is een heel bekend merk van drones. Ze hebben drones van alle prijsklassen alsook software die ermee gepaard gaat. Om deze nu te gaan programmeren zijn er enkele opties beschikbaar. Men moet wel rekening houden met de hardware van de drone bij het kiezen van één van deze opties. Sommige drones zullen namelijk niet voldoende uitgerust zijn met bepaalde hardware om een optie te gebruiken.

De eerste optie, en de optie die wij verder zullen gebruiken, is het gebruik maken van de mobile SDK van DJI. Deze SDK stelt de gebruiker in staat om een iOS of android app te creëeren op voorwaarde dat hij of zij voldoende voorkennis heeft van programmeren en de documentatie van de SDK grondig doorneemt. Indien deze app correct geprogrammeerd is, kan de gebruiker, de drone gaan besturen aan de hand van deze app. Het grootste nadeel is dat men heel voorzichtig moet zijn bij het gebruiken van zo een app wanneer de drone maar een beperkt aantal sensoren heeft. Indien de app, de drone opdraagt om ergens naartoe te vliegen dan zal de drone, zonder dit in vraag te stellen, daar naartoe vliegen. Dit kan gevaarlijk zijn omdat, wanneer de drone een obstakel niet "ziet", het ertegen kan vliegen. Dit kan voor ernstige schade of zelfs verwondingen zorgen. In onze usecase zouden deze obstakels apparatuur van het schip kunnen zijn. Er zijn reeds een groot aantal android apps alsook iOS apps beschikbaar op de respectievelijke app stores en deze apps bevatten een reeks van heel handige features. De app van DJI heeft bijvoorbeeld een "Active Tracking" feature die de drone in staat stelt om een object, dier of persoon de volgen ongeacht van waar het object, dier of de persoon naartoe beweegt. Ook is er een app genaamd Litchi die in staat is om de drone een voorgedefinieerd pad te laten volgen. Dit pad is gebasseerd op coördinaten en de drone zal van coördinaat naar coördinaat vliegen in een rechte lijn. We zullen deze apps later verder bespreken.

De tweede optie is het gebruiken van de Onboard SDK. Hiermee kan de gebruiker, de drone rechtstreeks gaan programmeren indien deze een computer aan boord heeft. Hier is echter nog een extra hardware requirement waaraan voldaan moet worden. Naast de Onboard computer moeten er namelijk voldoende sensoren aanwezig zijn zodat de drone autonoom kan bepalen waar het veilig is om te vliegen. Dit is in de nieuwere en dus duurdere modellen standaard het geval.

De derde en laatste optie die we zullen bespreken is de Windows SDK. Deze SDK kan ingezet worden bij het ontwikkelen van een Windows App. Hiervoor heb je opnieuw een nieuwer model nodig gezien hier opnieuw de vereiste van voldoende sensoren van kracht is. 

\paragraph{Andere drones}

Wat de andere drones betreft zullen deze variëren van drone tot drone alsook van merk tot merk. De duurdere drone die speciaal voor reddingsacties ontwikkeld werd is gebouwd met een aantal componenten van DJI en kan dus ook gebruik maken van hun SDK's. De drones die dat niet kunnen, hebben misschien hun eigen SDK's, kunnen via software geprogrammeerd worden of zijn ongeschikt voor onze usecase.
